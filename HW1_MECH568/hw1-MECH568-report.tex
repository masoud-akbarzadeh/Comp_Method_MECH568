%---------------------
% START OF PREAMBLE - do not delete!
%---------------------
\documentclass[12pt]{article}
\usepackage[pdftex]{graphicx}
\usepackage{amsmath}
\usepackage{verbatim}
\DeclareGraphicsRule{*}{mps}{*}{}

%==============================================================================
% Page layout
%==============================================================================

%------------------------------------------------------------------------------
%  Define the page dimensions.
%------------------------------------------------------------------------------
\setlength{\hoffset}{0.0in}
\setlength{\oddsidemargin}{0.0in}
\setlength{\evensidemargin}{0.0in}
\setlength{\textwidth}{6.75in}

\setlength{\voffset}{0in}
\setlength{\topmargin}{-.6in}
\setlength{\headheight}{12pt}
\setlength{\headsep}{12pt}
\setlength{\textheight}{9.5in}
\renewcommand{\baselinestretch}{1.0}
\renewcommand{\labelitemi}{-}

% writing the section number and the subsection number together
% and also the subsubsection in the form of 1.a
\renewcommand\thesubsection{\arabic{section}.\alph{subsection}}
\renewcommand\thesection{Problem \arabic{section}}
%------------------------------------------------------------------------------

%---------------------
% END OF PREAMBLE - do not delete!
%---------------------

\begin{document}

%---------------------
% make the title
%---------------------
\title{Mech 568 - Assignment 01 - Finite Difference}
\author{Masoud Akbarzadeh}
\date{\today}


\maketitle

%\newpage
%---------------------

%---------------------
% begin main text
%---------------------
\section{}\label{sec:problem-1}

\subsection{}\label{subsec:problem-1-a}
I used the 20-year wintertime North American surface temperature.
I chose six clusters for this analysis.
The analysis was done for 100 iterations but the results were the same for all the iterations more than 10.
\[ \frac{\partial^2 P}{\partial t^2} = c^2 \frac{\partial^2 P}{\partial x^2}  \]

\[ r = \frac{c^2 \Delta t^2}{\Delta x^2}   \]

Explicit method:

\[ P_{k+1,i} = r P_{k,i-1} + [2-2r] P_{k,i} + r P_{k,i+1} - P_{k-1,i}  \]

Implicit method:
\[ P_{k,i} - \frac{1}{2} P_{k-1,i} = -\frac{r}{2} P_{k+1,i-1} + \frac{(2r+1)}{2} P_{k+1,i} - \frac{r}{2} P_{k+1,i+1} \]


\subsection{}\label{subsec:problem-1-b}
Done.
\subsection{}\label{subsec:problem-1-c}
The six cluster centroids are shown in Figure 1.

\subsection{}\label{subsec:problem-1-d}
Done.
\subsection{}\label{subsec:problem-1-e}

To interpret the clusters in figure 1,
\begin{itemize}
    \item Cluster 1: It is warmer in Florida and Mexico region and colder on other parts of the US and Canada.
    \item Cluster 2: The temperature is the highest in the west and lower in the east.
    \item Cluster 3: It is warmer on the east side of the Rocky Mountains and colder on the west side.
    \item Cluster 4: The temperature is generally warm, and it is warmer in the north.
    \item Cluster 5: The temperature is lower in the North and warmer in the south.
    \item Cluster 6: It is cold in the south-eastern part of the US and mild in the north-western part.
\end{itemize}

The figure 2 shows the number of data points in each cluster.
It shows that cluster 2 has the most data points it means that non-harsh winters are more common with the temperature being mildly higher in the west.
and the fewest data points are in cluster 1 and 5.
These clusters show the temperature in what regions are most probable to correlate with each other.\\

\begin{figure}
\begin{center}
\includegraphics[width=\textwidth]{Figure_1.png}
\caption{The six clusturs for the analysis of temeprature anamolies in the winters of North America}{\label{fig: problem-2-c}}
\end{center}
\end{figure}



\end{document}


